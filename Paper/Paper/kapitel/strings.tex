\section{Strings}
Heutzutage ist es schon fast undenkbar und unmöglich ein Programm ohne \emph{Strings} zu
schreiben. Sie sind ein fester Bestandteil, bei so gut wie jeder höheren Programmiersprache,
geworden. Viele Programmierer sehen diese deswegen als etwas sehr leichtgewichtiges. Dies kommt
daher, dass Strings genauso behandelt werden wie primitive Datentypen, wie zum Beispiel \emph{int}
oder \emph{char}. Jedoch wäre es Fatal anzunehmen, dass Operationen mit „Strings“ genauso effizient
geschehen, wie bei den primitiven Datentypen. Dies zeigt auch das folgende Zitat:

\begin{zitat}
    Strings are convenient because they automatically grow as needed to hold their con‐ tents. By
    contrast, C library functions (strcat(), strcpy(), etc.) act on fixed-size character arrays.
    To implement this flexibility, strings are allocated dynamically. Dynamic allocation is
    expensive compared to most other C++ features, so no matter what, strings are going to show
    up as optimization hot spots \cite{OptimizedC++}
\end{zitat}

\subsection{Probleme mit Strings}
\emph{Strings} ziehen viele Probleme mit sich, jedoch sind die zwei Größten Probleme die
\emph{Strings} mit sich ziehen sind folgende:

\begin{itemize}
    \item \emph{Strings} werden auf dem \emph{Heap} gespeichert, was, wie im Kapitel
    \emph{Speicher Management} gezeigt ein sehr Teurer Prozess ist.
    \item Dadurch dass \emph{Strings} wie primitive Datentypen behandelt werden, verursachen
    \emph{Strings} viele Kopiervorgänge, was, wie im Kapitel \emph{Kopieren von Objekten} sehr
    suboptimal ist.
\end{itemize}

Um zu Demonstrieren, wie Problematisch diese zwei Punkte tatsächlich sind, soll die \emph{String}
klasse, die im laufe dieses Papers vorgestellt wurde um die \emph{+} und \emph{=} Operatoren
erweitert werden, wie im folgenden gezeigt:

\begin{lstlisting}[
    caption={String-Operatoren},
    label=lst:StringOperatoren,
    language=C++,frame=tlrb]
	String& operator=(const String& other) noexcept {
		printf("Copied via =\n");
		if (this == &other)
			return *this;

		delete[] m_Data;
		m_Size = other.m_Size;
		m_Data = new char[m_Size + 1];
		memcpy(m_Data, other.m_Data, m_Size + 1);
		return *this;
	}
	String& operator+(const String& other) noexcept {
		this->operator+(other.m_Data);
	}

	String& operator+(const char* other) noexcept {
		printf("Copied via +\n");
		String temp = *this;

		delete[] m_Data;
		int other_Size = strlen(other);
		m_Size = temp.m_Size + other_Size;
		m_Data = new char[m_Size + 1];

		int i = 0;
		for (int j = 0; j < temp.m_Size; ++j, ++i)
			m_Data[i] = temp.m_Data[j];
		for (int j = 0; j < other_Size; ++j, ++i)
			m_Data[i] = other[j];

		m_Data[m_Size] = 0;
		return *this;
	}
\end{lstlisting}

Damit die beiden Operatoren so funktionieren, wie sie sollen, muss der Speicher für \emph{m\_Data}
mittels \emph{new} vergrößert werden. Darauf hin müssen die Daten, entweder mit Standard Funktionen
wie \emph{memcpy} oder mit simplen \emph{for-schleifen}, kopiert werden.
\newline
\newline
Da nun die beiden Operatoren zur Verfügung stehen, kann nun mit der folgenden Sequenz, die
Problematik zur Schau gestellt werden. Hierfür wird außerdem noch der \emph{new} Operator
überschrieben um später zu überprüfen, wie oft die Funktion aufgerufen wurde.

\begin{lstlisting}[
    caption={String-Operationen},
    label=lst:StringOperationen,
    language=C++,frame=tlrb]
static uint64_t Allocations = 0;

void* operator new(size_t size) {
	++Allocations;
	return malloc(size);
}

void PrintString(String string) {
	std::cout << string << std::endl;
}

int main(int argc, char** argv)
{
	String demo = String("Hello") + " World" + "!";
	PrintString(demo);
	std::cout << "New called " << Allocations << " times" << std::endl;
	std::cin.get();
}
\end{lstlisting}
Wird die Sequenz betrachte, dann ist dies nichts Weltbewegendes, jedoch resultiert diese Sequenz
schließlich in diesen Output:

\begin{lstlisting}[
    caption={Ausgabe des String-Programms},
    label=lst:ArrayAdressen,
    language=C++]
Created
Copied via +
Copied
Copied via +
Copied
Copied
Copied
Hello World!
New called 7 times
\end{lstlisting}

Tatsächlich wurde für dieses doch sehr simple Programm Sieben mal der \emph{new} Operator
aufgerufen. Eine Optimierungsansatz für dieses Problem, sind \emph{Move Semantiken}, die im
späteren Verlauf dieses Papers noch besprochen werden, jedoch ist dieses Verhalten aus Sicht der
Performance inakzeptabel und der Einsatz von \emph{Strings} sollte auf ein nötigstes beschränkt
werden.
