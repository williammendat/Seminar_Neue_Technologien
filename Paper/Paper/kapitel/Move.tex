\section{Move-Semantiken}\label{sec:move}
Wie im Kapitel \emph{\nameref{sec:tempobj}} schon gezeigt wurde, führen temporäre Objekte
unnötigerweise
dazu, das Objekte nur erzeugt werden, um dann die Daten an einer anderen Stelle weiterzugeben.
Dieses Verhalten führt bei einem komplexeren Objekt, wie zum Beispiel einem String dazu, dass
Speicher auf dem \emph{Heap} bereitgestellt werden und somit zur Lasten der Performance geht.
\newline
\newline
Mit der Veröffentlichung von C++11 wurden \emph{Move-Semantiken} eingeführt, die es erlauben, dass
die Daten von einem Objekt, von einer Variable gestohlen werden können. Anstatt das Objekt
erst erzeugt wird, um dann die Daten zum Ziel zu kopieren, wird das Objekt nur noch erzeugt und
die \emph{Pointers} werden umgestellt \cite{VisualisingC++}[vgl.]. Um von den Features der
\emph{Move-Semantiken} Gebrauch machen zu können, muss das Objekt einen \emph{Move-Constructor}
bereitstellen. Dieser \emph{Move-Constructor} erwartet als Übergabe Parameter eine sogenannte
\emph{rvalue-reference}, aus welche die Daten dann letztendlich gestohlen werden können. Eine
\emph{rvalue-reference} kann, wie der Name schon sagt, eine \emph{rvalue} sein oder auch ein Objekt
welches schon vorher erzeugt wurde, jedoch muss dieses Objekt dann mittels \emph{std::move()}
gekennzeichnet werden.
\newline
\newline
Um zu demonstrieren, wie ein solcher \emph{Move-Constructor} aussehen könnte und wie später damit
hantiert werden kann, soll zunächst die eigen-implementierte \emph{String} klasse mit einem solchen
\emph{Move-Costructor} erweitert werden:

\begin{lstlisting}[
    caption={String \emph{move construtor}},
    label=lst:MoveKonstruktor,
    frame=tlrb]
class String {
	[...]
   String(String&& other) {
        printf("Moved\n");
        m_Size = other.m_Size;
        m_Data = other.m_Data;
        other.m_Data = nullptr;
        other.m_Size = 0;
    }
	[...]
};
\end{lstlisting}

Wie zu sehen ist, wird eine \emph{rvalue-reference} mittels \emph{\&\&} gekennzeichnet. Zudem ist
auch zu erkennen, dass es hier nicht vonnöten ist, mittels \emph{new} Speicher auf dem \emph{Heap}
anzulegen. Lediglich der \emph{Pointer} wird umorientiert. Wichtig hierbei ist, dass der
\emph{Pointer} von \emph{other.m\_Data} auf nullptr gesetzt wird, sonst entsteht die
Situation, dass beim Löschen von der \emph{rvalue-reference} die angelegten Daten auf dem
\emph{Heap} auch gelöscht werden.
\newline
\newline
Im Kapitel \emph{\nameref{sec:strings}} wurde ein Beispiel (Listing \ref{lst:StringOperationen}
\emph{\nameref{lst:StringOperationen}}) vorgeführt, welches in diesem Kapitel auch wieder zum
Einsatz kommen soll. Nur dieses Mal unter der Verwendung von \emph{Move-Semantiken}. Durch den
\emph{Move-Construtor} und das Kennzeichnen der Objekte als \emph{movable} Objekte resultiert
das gleiche Programm in folgendem Output:

\begin{lstlisting}[
    caption={Ausgabe des String-Programms mit Move-Semantiken},
    label=lst:StringProgrammMitMove,
    language=bash]
Created
Copied via +
Moved
Copied via +
Moved
Moved
New called 3 times
\end{lstlisting}

Lediglich durch die Verwendung von diesem Feature hat sich die Zahl an \emph{new} Aufrufen halbiert.
Wird dabei ein großes Projekt betrachtet, mit mehreren 1000 solcher Operationen, kann alleine
dadurch viel Performance gewonnen werden.