\section{Einleitung}\label{sec:einleitung}
\lettrine{B}{ei} dem Gedanken an die Programmiersprache C++ erschauern viele Programmierer, sei
es nun der blutige Anfänger, der noch nie etwas mit Programmierung zu tun hatte, oder der
langjährig erfahrene Embedded Programmierer, der lieber auf seine bewährte Programmiersprache C
zurückgreift. Dabei stellt sich die Frage, woher die Abneigung gegen diese Sprache kommt, wenn
Sie doch Perfekt scheint mit Ihrer nähe zur Hardware und den Vorteilen einer hohen
Programmiersprache. Gut, die Antwort darauf ist simpel, denn C++ ist ein Fluch und ein Segen
zugleich. C++ ist die momentan mächtigste Programmiersprache und in keiner anderen
Programmiersprache bekommt der Programmierer so viele Freiheiten, jedoch ist dies wie alles im
Leben nicht um sonst, denn mit vielen Freiheiten kommt auch viel Verantwortung.
\newline
\newline
Dieses Paper soll sich jedoch nicht mit dem sicheren Programmieren von C++ beschäftigen, sondern
eher damit, wie C++ ausgenutzt werden kann, um sehr performante Software zu schreiben. Tatsache
ist nämlich, dass genau so Performanter, wenn nicht sogar noch performanteren Code geschrieben
werden kann wie in C. Jedoch um dies zu erreichen, sollte sich an die ursprünglichen Prinzipien
von C++ gehalten werden. Eines der ersten und zudem auch wichtigsten Prinzipien ist das
\emph{zero overhead abstraction} Prinzip beziehungsweise, nicht für das zu Bezahlen, was nicht
gebraucht wird \cite{HandsOn}[vgl.]. Unterstrichen wird dies noch mit dem Zitat:

\begin{zitat}
    Technically, C++ rests on two pillars. A direct map to hardware (initially from C) and
    zero-overhead abstraction in production code (initially from Simula where it wasn´t zero
    overhead). Depart from those and the language is no longer C++ \cite{ISOC++}.
\end{zitat}


		
		
