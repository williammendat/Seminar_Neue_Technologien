\section{Templates}\label{sec:templates}
\begin{zitat}
    C++ templates allow functions and classes to operate on generic types to increase code
    reusability without sacrificing performance. \cite{TemplatesVorteil}
\end{zitat}
\emph{Templates} bieten einige Vorteile, wie auch im Zitat am Anfang gezeigt wurde, jedoch ist der
Fakt, dass \emph{Templates} für generische Typen benutzt werden können, nicht das einzig
besondere an denen.
\newline
\newline
Da \emph{Templates} zur Compilezeit und nicht zur Laufzeit erzeugt werden, können diese dafür
gebraucht werden, um Berechnungen zur Compilezeit durchzuführen \cite{HandsOn}[vgl.]. So kann zum
Beispiel die Summe einer konstanten Zahl zur Compilzeit wie folgt berechnet werden:
\begin{lstlisting}[
    caption={Summe mittels \emph{Templates}},
    label=lst:SummeTemplate,
    language=C++,frame=tlrb]
template <unsigned int Base>
struct Sum
{
	static const unsigned long long result = Base + Sum<Base - 1>::result;
};

template<>
struct Sum<1>
{
	static const unsigned long long result = 1;
};

int main(int argc, char** argv)
{
	std::cout << Sum<5>::result << std::endl;
	std::cin.get();
}
\end{lstlisting}

Wichtig dabei ist allerdings, dass die Variable \emph{result} als \emph{static} variable
deklariert wird, sonst kann diese Berechnung so nicht funktionieren.
\newline
\newline
Obwohl \emph{Templates} augenscheinlich Vorteile bieten, werden diese oft gemieden und in manchen
Projekten sogar komplett verboten. Dies kommt daher da \emph{Templates}, wenn es komplexer wird,
sehr schwer zu verstehen sind und wenn es zu einem Fehler kommt, meist sehr schwer ist, diesen zu
finden.