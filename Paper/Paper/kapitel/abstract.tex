\begin{abstract}
	Schon seit mehreren Jahrzehnten begleitet uns Software und stellt somit einen signifikanten Bestandteil unseres Lebens dar. In beinahe jeder Lebenssituation heutzutage findet sich Software wieder, sei es nun in den unzähligen Mikrocontrollern eines modernen Autos, welches ein hightech Computer auf rädern darstellt oder ein Videospiel, um von einem langen Arbeitstag abzuschalten. Die Anforderungen an Applikationen wachsen dabei stetig, weshalb neben logischer Korrektheit vermehrt auch Geschwindigkeit von übergeordneter Bedeutung ist. Vor allem im Embedded Bereich bekommt die Geschwindigkeit eines Programms noch mal eine viel größere Bedeutung, da hier auch meist in Lebens gefährdeten Bereichen hantiert wird. Werden die letzten Jahrzehnte betrachtet, zeigt sich, dass die vorherrschende  Programmiersprache im Embedded Bereich C ist. Jedoch zeigt sich auch, dass ein Trend zur Programmiersprache C++ entwickelt wurde. 
\newline
Dieses Paper befasst sich mit C++, einer Programmiersprache, die ursprünglich von Bjarne Stroustrup im Jahr 1979 als eine Erweiterung von C entwickelt wurde. Genauer soll darauf eingegangen werden, wie mit C++ sehr performant programmiert werden kann. Dabei soll gezeigt werden, wie mit einfach Ticks die Performance verbessert werden kann.
\end{abstract}