\section{Runtime Type Identification}\label{sec:rtti}
Runtime type Indentification kurz \emph{RTTI}, kann genutzt werden, um sicher zwischen
verschiedenen Typen in einer Klassenhierarchie zu \emph{casten}. Dabei verwendet \emph{RTTI}
Metadaten, die einer Klasse hinzugefügt werden, die dann zur Laufzeit verwendet werden können.
Mittels dem \emph{$dynamic\_cast$} Operator kann überprüft werden, ob zwischen einer Basisklasse
und einer abgeleiteten Klasse gecastet werden kann. Im Folgenden wird ein klassisches Beispiel
für die Verwendung von \emph{$dynamic\_cast$} demonstriert:
\begin{lstlisting}[
    caption={Demonstration von \emph{dynamic\_cast}},
    label=lst:Dynamic-Cast,
    language=C++,frame=tlrb]
class Stringbuilder : public String {
public:
	Stringbuilder(const char* string)
		:String(string) { }

    void PrintString() override {
    	std::cout << *this
       	<< " printed with Stringbuilder" << std::endl;
    }
};

int main(int argc, char** argv){
	String demo = new Stringbuilder("Demo");
	Stringbuilder* demoBuilder =
	dynamic_cast<StringBuilder*>(demo);

	if(!demoBuilder){
		//handle Error
	}
}
\end{lstlisting}

Hier kann nun geprüft werden, ob der \emph{cast} erfolgreich war, indem die Variable
\emph{demoBuilder} auf \emph{null} geprüft wird und wenn dem der Fall ist, dann war der
\emph{cast} nicht erfolgreich und es kann darauf reagiert werden.
\newline
\newline
Jedoch kommt es nun zu folgendem Problem: \emph{RTTI} fügt aus Performance Sicht unnötigen
\emph{overhead} hinzu und die Überprüfung bei \emph{$dynamic\_cast$} findet zur Laufzeit statt.
Wie schon in der Einleitung erwähnt, ist das \emph{zero overhead prinzip} eines der, wenn nicht
sogar das wichtigste Prinzip, welches hier klar verletzt wird. In dem Sinne sollte, sofern
möglich, \emph{RTTI} ausgeschaltet werden und auf Alternativen zurückgegriffen werden. Hier
existiert kein wissenschaftlicher Grundsatz, jedoch eine Möglichkeit, zumindest
\emph{$dynamic\_cast$} zu ersetzten, wäre die Verwendung einer \emph{Virtuellen Funktion}.