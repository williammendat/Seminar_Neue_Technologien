\section{Constant expressions}\label{sec:constexpr}
Wie im letzten Kapitel \emph{\nameref{sec:templates}} zu sehen war, ist es möglich Berechnungen
zur Compilezeit durchzuführen, jedoch machen es die schlechte Lesbarkeit und die unverständlichen
Fehlermeldungen ziemlich schwer, mit \emph{Templates} zu Hantieren. Aufgrund von dieser Tatsache,
kamen mit der Veröffentlichung von \emph{C++11} auch die \emph{constexpr} Funktionen. Eine solche
Funktion beginnt mit dem Keywort \emph{constexpr} und könnte wie folgt aussehen
\cite{HandsOn}[vgl.]:

\begin{lstlisting}[
    caption={Summe mittels \emph{constexpr}},
    label=lst:SummeConstexpr,
    language=C++,frame=tlrb]
constexpr uint32_t Sum(uint16_t value) {
	return value <= 1 ? 1 : (value * Sum(value - 1));
}

int main(int argc, char** argv)
{
	std::cout << Sum(5) << std::endl;
	std::cin.get();
}
\end{lstlisting}

So konnte erreicht werden, dass der gleiche Effekt wie, in Listing \ref{lst:SummeTemplate} \nameref{lst:SummeTemplate},
geschaffen wurde und das sehr leicht und leserlich. \emph{Constant expression} ist ein sehr
mächtiges Feature von C++ und sollte so oft wie nur möglich verwendet werden.